\documentclass{article}
\usepackage{times}
\usepackage[utf8]{inputenc}

\begin{document}
	
	
	\title{ORSI 1. beadandó feladat - dokumentáció} 
	\author{Kovács Bálint - FANW4Z}  %\texttt formats the text to a typewriter style font
	\date{\today}  %\today is replaced with the current date
	\maketitle
	
	\section{Feladat}
	A bemenet -- \texttt{input.txt} -- első sorában tartalmazza a szóközzel elválasztott N és M pozitív egészeket -- ahol következő N sorában pedig egy-egy neptunkód, játéknév és percben eltöltött idő (egész szám) szóközzel elválasztott hármasát:
	
	\begin{verbatim}
	N M
	NEPTUN_1 gameazon perc   - az 1. válasz adatai
	NEPTUN_2 gameazon perc   - az 2. válasz adatai
	...
	...
	...
	NEPTUN_N gameazon perc   - az N. válasz adatai
	
	//Egy lehetséges konkrét sor pl:
	//BATMAN WoW 1500
	\end{verbatim}
	
	
	A főfolyamat olvassa be az adatokat, indítson M gyerekfolyamatot, majd minden gyerekfolyamathoz társítson egy-egy játékhoz tartozó adathalmazt. (A játékok száma pontosan M, így minden játék adatát párhuzamosan kell kiszámolni.) A gyerekfolyamatok dolga megállapítani, hogy az adott játékkal mennyi időt töltöttek összesen (percben) a válaszadók, valamint az átlagos játékidő kiszámítása az adott játékhoz (egész percre lefelé kerekítve - floor(..) ). Ezt a két adatot küldje vissza a szülőfolyamatnak.
	
	A főfolyamat ezek után a játékokhoz tartozó, összesen és átlagosan eltöltött időt szóközzel elválasztva a JÁTÉKOK azonosítója ALAPJÁN BETŰRENDBEN írja az output.txt kimeneti fájlba.
	
	\begin{verbatim}
	//egy példa sor az output fájlból:
	//WoW 6000 573
	\end{verbatim}
	
	\section{Felhasználói dokumentáció}
	
	\subsection{Környezet}
	A program több platformon futtatható, nincsen dinamikus függősége. Telepítésre
	nincs szükség, elegendő a futtatható állományt elhelyezni a számítógépen.
	\subsection{Használat}
	A program elindítása egyszerű, mivel nem vár parancssori paramétereket, így
	parancssoron kívül is lehet futtatni. A fájl mellett kell elhelyezni az \texttt{input.txt}
	fájlt, melyet feldolgoz és az eredményt az \texttt{output.txt} nevű fájlba írja, a betűrend  alapján.
	Egy lehetséges bemenetet tartalmaz a mellékelt \texttt{input.txt} tesztfájl. Saját
	bemeneti fájlok esetén a sorok és a játékok számának pontos megadása nem fontos, elég ha az elején kihagyunk egy sort. Viszont  a helyes működéshez szükséges a többi sor megfelelő formátuma, és hogy ne legyen több üres sor a dokumentumban.
	
	\section{Fejlesztői dokumentáció}
	
	\subsection{A megoldás módja}
	A kódot logikailag két részre bonthatjuk, egy fő és több alfolyamatra. A főfolyamatot
	a \(start()\) függvény fogja megvalósítani, ez fogja beolvasni az input fájl tartalmát,
	azt csoportosítja és nemszabályos mátrixba rendezi. Minden alfolyamathoz a mátrix egy sorát
	társítjuk majd, ezekkel számolnak. Az eredményt a főfolyamat fogja beleírni
	a kimeneti fájlba.
	
	\subsection{Implementáció}
	Az Erlang nyelvi elemeit kihasználva \(orddict\) adatszerkezet fogja a különböző szálakon futó függvények visszatérési értékét tárolni. A szükséges N folyamatot a beépített \(spawn()\) függvény segítségével fogjuk párhuzamosan indítani a \(start\_process()\) függvényben, majd minden szál az \(addAndAvg()\) függvényt fogja a szótár hozzá
	tartozó adatára végrehajtani. Az imént említett függvény a számtani közép alapján számol átlagot, azaz
	\\ \( x_{avg} = \frac{1}{n} \sum_{i=1}^{n} x_{i} \) -t számolja ki, és a ezzel az elemek összegét, majd ezekkel az értékkel tér vissza. A teljes implementáció egyetlen forrásfájlba szervezve, a \texttt{orsi\_bead1.erl} fájlban található.
	
	\subsection{Fordítás}
	A program forráskódja az \texttt{orsi\_bead1.erl} fájlban található. A program fordításához
	követelmény Erlang fordítóprogram megléte a rendszeren. Fordítás után az \(orsi\_bead1\) modul \(start()\) folyamatának meghívásával futtathatjuk a programot.
	
	\subsection{Tesztelés}
	A programot szélsőséges inputokkal teszteltem. Ezek:
	\begin{itemize}
		\item \texttt{empty.txt} - bemenet 0 adatsorral
		\item \texttt{9999.txt} - sok adat, a futási idő elfogadható
		\item \texttt{onlyDota.txt} - egyféle játék 
	\end{itemize}
	
\end{document}