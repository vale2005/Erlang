\documentclass{article}
\usepackage{times}
\usepackage[utf8]{inputenc}
\usepackage{graphicx}

\begin{document}
	
	
	\title{ORSI 3. beadandó feladat - dokumentáció} 
	\author{Kovács Bálint - FANW4Z}  %\texttt formats the text to a typewriter style font
	\date{\today}  %\today is replaced with the current date
	\maketitle
	
	\section{Feladat}
	A bemeneti fájlok egyike, az \texttt{input\_matrices.txt} első sorában egy M pozitív egész olvasható, ennyi lineáris transzformációt kell elvégezni az objektumokat reprezentáló vektorokon, míg a következő 4xM sorban egy-egy 4x4es mátrix sorfolytonos reprezentációja található szóközökkel elválasztva. (4 egymás utáni sor jelent egy mátrixot.) A másik fájl, az \texttt{input\_points.txt} tartalmazza a pontok listáját, amikre alkalmazni kell az előbbi mátrixokkal való szorzást. Ennek első sora egy egész szám, N, ez után található N sor egy-egy 3 dimenziós vektor koordinátáit írja le (szintén szóközzel tagolva).
	
	Egy lehetséges \texttt{input\_matrices.txt} fájl: (Az alábbi mátrix-transzformációk pl. a 2x-es méretezés, 90-os forgatást az X tengely körül, majd a (3;-2;4) vektorral való eltolást jelentik.)
	
	\begin{verbatim}
	3
	2 0 0 0
	0 2 0 0
	0 0 2 0
	0 0 0 1
	1 0 0 0
	0 0 -1 0
	0 1 0 0
	0 0 0 1
	1 0 0 3
	0 1 0 -2
	0 0 1 4
	0 0 0 1
	\end{verbatim}
	Egy lehetséges \texttt{input\_points.txt} fájl:
	
	\begin{verbatim}
	8
	1 1 1
	1 1 -1
	1 -1 1
	1 -1 -1
	-1 1 1
	-1 1 -1
	-1 -1 1
	-1 -1 -1
	\end{verbatim}
	A számítást a későbbiekben megkönnyíti, ha a bemenet során a 3D-s vektorokat 4 dimenzióban reprezentáljátok, a negyedik koordinátát 1-re inicializálva. A mátrixszal való szorzás így lényegesen könnyebbé válik.
	
	A feladatban az adatcsatorna tételére visszavezetve kell megoldani a kitűzött problémát!
	
	A főfolyamat dolga, hogy beolvassa a mátrixokat, majd M threadet létrehozva, s azokat egy-egy transzformációnak megfeleletetve átadja nekik az megfelelő mátrixokat. A pontokat az első mátrixot reprezentáló szálnak kell elküldeni, majd az utolsó leképezést megvalósító gyerektől fogadja a megfelelően transzformált vektorokat. Ezek után írja soronként az \texttt{output.txt} fájlba az így kapott eredményt.
	
	Az indított szálak feladata, hogy fogadják a vektorokat a megelőző gyerektől (az első transzformáció esetén a mastertől), elvégezzék a mátrix-vektor szorzást, majd továbbítsák az így kapott részeredményt a következő folyamatnak. (Az utolsó számítás esetén a masternek.) A teljes memóriahasználat csökkentése érdekében használhattok külön szálat a beolvasás és kiírás elvégzésére.
	
	A dokumentációban mindenképp szerepeljen a visszavezetés mikéntje, azaz a megfeleltetés az absztrakt programhoz! A fejlesztői fejezetben szeretnénk mérésekkel alátámasztva látni, hogyan is skálázódott a program a különböző méretű bemenetek esetén! Egy lehetséges output.txt fájl:
	
	\begin{verbatim}
	5 -4 6
	5 0 6
	5 -4 2
	5 0 2
	1 -4 6
	1 0 6
	1 -4 2
	1 0 2
	\end{verbatim}

	
	\section{Felhasználói dokumentáció}
	
	\subsection{Környezet}
	A program futtatásához Erlang shell szükséges. 
	
	\subsection{Használat}
	Fordítsunk \(c(matrices)\) paranccsal, majd futtathatjuk a \(matrices:start()\) függvényhívással. A fájl mellett kell elhelyezni az \texttt{input\_points.txt} fájlt, mely a transzformálandó pontokat, és az \texttt{input\_matrices.txt} fájlt, mely a transzformáló mátrixokat tartalmazza. Az eredményt az \texttt{output.txt} nevű fájlba írja a program, bementi sorrend szerint. Saját bemeneti fájlok esetén a mátrixok számának pontos megadása nem fontos, a pontoké viszont igen. Viszont  a helyes működéshez szükséges a többi sor megfelelő formátuma(ne legyen szóköz a sorok végén), és hogy ne legyen több üres sor a dokumentumban.
	
	\section{Fejlesztői dokumentáció}
	
	\subsection{A megoldás módja}
	Egy mellékfolyamat beolvassa az \texttt{input\_points.txt} fájlt, míg a főfolyamat az \texttt{input\_matrices.txt}-vel teszi ugyanezt. Létrehozza a mátrixokat, és a pipeline-okat, elküldve a célfolyamat azonosítóját. A folyamatok végzik a transzformációt. A transzformált pontokat kiírjuk a folyamat végén. 
	
	\subsection{Implementáció}
	A feladatot adatcsatorna tétel szerint oldjuk meg:
	\begin{itemize}
		\item $D =< d_{1}, ..., d_{n} >$ - a bemenő pontok halamaza
		\item $x_{i}$ változatlan
		\item $F=M_{m}\times...\times M_{0}$ - a mátrixszorzások egymásutánja
		\item $f_{0}(v) = SKIP$ - nincs kezdeti művelet
		\item$f_{i}(v) = M_{i}\times v (i \in 1..n)$ - az \(i\)-edik mátrixszal való szorzás
	\end{itemize}
	
	\subsection{Fordítás}
	Fordítsunk \(c(matrices)\) paranccsal, majd futtathatjuk a programot \(matrices:start()\) függvényhívással.
	
	\subsection{Tesztelés}
	Tesztesetek és eredmények
	\begin{itemize}
		\item 20000 pont, 3 mátrix $\,\to\,$ 0,586s
		\item 20000 pont, 6 mátrix $\,\to\,$  0,689s
		\item 20000 pont, 12 mátrix $\,\to\,$  0,754s
		\item 40000 pont, 3 mátrix $\,\to\,$   1,070s
		\item 40000 pont, 6 mátrix $\,\to\,$   1,199s
		\item 40000 pont, 12 mátrix $\,\to\,$   1,450s
	\end{itemize}
	Az tesztekben a mátrixok száma elhanyagolható a pontok számához képest. Így a mérések igazolják, hogy N+M-es az algoritmus: a pontok számának duplázásával a futási idő valóban a duplája lett, a mátrixok számának növekedésével pedig az idő körülbelül lineárisan nő.
	
\part{\end{document}}